\documentclass[11pt,a4paper]{article}
\usepackage[
    left=0.73in,
    right=0.73in,
    top=.8in,
    bottom=.50in,
    paperheight=11in,
    paperwidth=8.5in
]{geometry}

\begin{document}
% Header
{\parindent0pt
ECE562 Lab 5\\
Interrupt Handling\\
Part 1\\
\today\\
Christopher Chin, Computer Science\\

\textbf{Objectives:}

The goal of this lab is to become more familiar with interrupts.\\

\textbf{Results:}

The green LED flashes 1 time per second because you set the real time clock interrupt to occur 1000 times per second.
The real time clock interrupt calls an interrupt handler that toggles the LED if a counter reaches RTIMAX which is 500.
The counter is incremented for each call so 500 calls must occur to toggle the LED.
This causes the LED to toggle every half second, which results in a flash every 1 second.

The green LED is unaffected by the key entering and display process because interrupts cause the program control flow to move to an interrupt handler.
The interrupts are handled every millisecond and must be serviced.
\\~\\
The red LED flashes continuously because the IRQ pin interrupts are being called constantly.
It is being called constantly because it is level triggered.
The interrupt causes the IRQSERV routine to be called continuously.
During IRQSERV the I-flag is set so another interrupt cannot interrupt it.
The RTI command unsets the I-flag.

The green LED stops flashing because the IRQ pin interrupts have a higher priority than the RTI interrupts.
Whenever the controller tries to service an RTI interrupt it receives an IRQ pin interrupt and instead runs the IRQSERV routine

Key presses are ignored because the interrupt is being constantly called so there is no time to run the main program.
The IRQSERV routine contains a delay which simply wastes CPU time.
The RTI handler on the other hand returns to regular program flow quickly and gets called one millisecond later.
\\~\\
Because IRQ interrupts are edge triggered it only occurs when the IRQ pins change levels.
Since the IRQ pins aren't changing, regular operation follows.

As I insert the wire, the red LED flashes twice due to electrical noise.

As the wire is removed, the red LED flashes due to vibrations causing the contact to break and reattach.
}
\end{document}
